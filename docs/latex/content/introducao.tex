\chapter{Introdução}
\label{chap:introducao}
\section{Contextualização}
  O seguinte projeto faz uma paradia em relação as armadilhas presetes nos filmes da franquia Jogos Mortais (Saw), em especifico a armadilha do \textit{Reverse Bear Trap} (Armadilha do Urso Reverso). 
Esta armadilha é uma das mais icónicas da franquia, e consiste numa máscara metálica que é colocada na cabeça da vítima, a qual está presa por um mecanismo que, quando ativado, faz com que a máscara se feche rapidamente. 
 
  Nesse projeto em especificoa vítima se encontra amarrada a uma cadeira, com visão apenas a um display digital com quatro números e acesso a quatro botões presentes no braço da cadeira ao alcançe de seus dedos. Ela possui um tempo limitado para resolver o puzzle, que consiste em pressionar os botões na ordem correta e na quantidade de vezes exibida pelo display. Se ela não conseguir resolver o puzzle dentro do tempo estipulado, a armadilha é ativada. 

O objetivo do projeto é desenvolver um sistema que simule essa armadilha usando conceitos de automação e controle.

