\chapter{Descrição}

\section{Puzzle}
O puzzle consiste em quatro botões que devem ser aperdados em sequência e na quantidade de vezes exibida no display. Cada botão está associado a um número de 1 a 4, e o display mostra uma sequência de quatro números que indicam a ordem correta para pressionar os botões. Por exemplo, se o display mostrar "2 4 1 3", a vítima deve pressionar o botão 2 duas vezes, o botão 4 quatro vezes, o botão 1 uma vez, e o botão 3 três vezes, nessa ordem específica.

\section{Tempo}
A vítima tem um tempo limitado para resolver o puzzle, que é exibido por outro display. Cada vez que a vítima pressiona um botão de forma incorreta, o tempo restante é reduzido em 10 segundos. Se o tempo chegar a zero antes que o puzzle seja resolvido, a armadilha é ativada. 

\section{Erros}
A cada vez que a vitima comete um erro, além do desconto de 10 segundos no tempo restanta, o valor de quatro digitos no display é alterado de forma aleatória, ou seja, o puzzle é alterado. Segue os erros possíveis:

\begin{itemize}
    \item Pressionar mais de um botão ao mesmo tempo
    \item Pressionar botão fora da sua ordem correspondente
    \item Pressionar o botão novamente após a bobina correspondente ter sido ativada
\end{itemize}

